\chapter{l'ouverture de 2\C}

\begin{multicols}{2}


L'ouverture de 2\C est très précise, elle va de 12 à 14H. L'enchère promet 5 cartes à \C et dénie 4 cartes à \P. Mis à part en troisième position, il n'est pas recommandé d'ouvrir les mains de 11H. Comme toute enchère faite au niveau de deux, la performance est dans la précision.

Avec une main 5-3-3-2, on ouvre systématiquement de 1\NT. Avec une main 5-4-2-2, on ouvrira parfois de 1\NT. Dans une main de 12-13H, dès qu'il y a un honneur autre que l'As dans chaque doubleton, on ouvre de 1\NT. Avec une main de 14H, il est possible d'ouvrir de 1\NT avec une main particulièrement laide, typiquement avec RD et Rx comme doubletons par exemple.

Le plus souvent, la main sera donc 5-4-3-1 ou 6-3-2-2. Mais elle peut comporter une mineure 5\ieme ou 6\ieme et les cœurs peuvent très longs.

Toutes les réponses de 2\NT à 3\P sont naturelle et limite de manches (y compris le soutien à 3\C qui n'est pas un barrage). Les enchères de 3\NT et 4\C sont conclusives. Les réponses de 4\T et de 4\K n'existent pas. \textit{On peut convenir que ce sont des splinters}.

Un relai forcing de manche à 2\P existe dont l'utilité principale est de retrouver un fit à pique mais qui peut aussi démarrer des séquences de chelem.

\enchbox{Ouverture de 2\C}
{
2\P &12H+& Interrogative à \P \\
2\NT & 11HL & Naturel \\
3\T  & 9-11H & 6 belles cartes \\
3\K & 9-11H & 6 belles cartes \\
3\C & 11DH & propositionnel \\
3\P & 9-11H & 6 belles cartes \\
3SA && Conclusion \\
4\C && Conclusion \\
4\P && Conclusion \\
4\NT && Blackwood brutal \\
}

\end{multicols}

\titre{2\C--2\P}
\begin{multicols}{2}


Le répondant peut relayer à 2\P pour plusieurs raisons. Il peut chercher à localiser le singleton pour éviter un mauvais contrat à \NT.
Avec 5 cartes à \P, il peut aussi rechercher un fit. Plus rarement, il souhaite jouer un chelem.

L'ouvreur dévoile en premier son nombre de cartes à \P.

Sur toute autre redemande que 2\NT, le répondant connaît la distribution de l'ouvreur à une carte près. Sur la redemande à 2\NT, il reste une incertitude sur la présence ou non d'un singleton mineur ou d'une mineur 4\ieme.

Quoi qu'il en soit, la situation est forcing de manche, des enchères naturelles permettront de trouver la meilleure manche lorsque 3\NT est inacceptable.

\noindent
\enchbox{2\C--2\P}
{
2SA && 5 cartes à \C et 3 cartes à \P \\
3\T && 4 cartes à \T et 0-2 cartes à \P \\
3\K && 4 cartes à \K et 0-2 cartes à \P \\
3\C && 6 cartes à \C et 0-2 à \P   \\
3\P && 6 cartes à \C et 3 cartes à \P \\
3SA && (n'existe pas)\\
4\T && 6 cartes \\
4\K && 6 cartes \\
4\C && 8 cartes \\
}


\end{multicols}
\titre{2\C--2\P--2\NT}

\begin{multicols}{2}
L'ouvreur vient d'annoncer qu'il a 3 cartes à \P. Le plus souvent, le répondant aura 5 cartes lui-même et sera en mesure de conclure.

Les enchères de 3\NT et de 4\P sont des conclusions, de même que l'enchère bizarre de 4\C (le répondant cherchait sans doute un chelem miracle).

Comme la situation est forcing de manche, les enchères de 3\C et de 3\P fixent l'atout pour explorer le chelem.

Les enchères de 3\T et de 3\K sont les plus difficiles. Elles sont naturelles bien sur mais peuvent provenir de mains très différentes.
Le répondant peut craindre une entame dans l'autre mineure où l'ouvreur pourrait avoir un singleton. Il peut lui-même avoir un singleton pique et souhaite vérifier que les 3 cartes en face sont solides. Plus rarement, il peut avoir envie de jouer un chelem dans sa mineure.
Quoi qu'il en soit, l'ouvreur nomme 3\NT s'il garde correctement les deux couleurs restantes. Avec une seule couleur gardée, il la nomme (sur 3\K, il dit 3\C quand il garde les \T à cause du classique problème des carreaux). S'il est faible dans les deux couleurs restantes, le contrat de 3\NT est exclu, l'ouvreur se contentera de soutenir la mineure.


\end{multicols}
\titre{2\C--2\P--3\T}
\begin{multicols}{2}
Rares sont les cas où le répondant ne peut pas conclure. Il peut montrer 6 cartes à \P en situation forcing en disant 3\P mais comme cette enchère ne fixe pas l'atout, ce n'est pas une tentative de chelem. Par contre, les enchères de 3\C et 4\T fixent l'atout et déclenche l'exploration de chelem.

La redemande à 3\K est naturelle et il est temps pour l'ouvreur de se rappeler que rien n'a été promis à \P, une conclusion à 3\NT ne peut se faire qu'avec l'arrêt pique. S'il ne sait pas quoi dire, l'enchère de 3\P devient une quatrième forcing.

La séquence 2\C--2\P--3\K est similaire.



\end{multicols}
\titre{2\C--2\P--3\C}
\begin{multicols}{2}

Pas beaucoup de place pour explorer. Toutefois, le répondant a tous les éléments pour conclure. Attention, les redemandes à 4\T et 4\K ne sont pas des contrôles mais des enchères naturelles ! S'il veut jouer un chelem à \C, le répondant fera un Blackwood brutal. S'il veut jouer à \P, il commencera par les exprimer au niveau de 3.

\end{multicols}

\titre{2\C--2\P--3\P}
\begin{multicols}{2}

Dans cette séquence, les redemandes à 4\T et 4\K sont naturelles et dénient un intérêt pour les piques.

La redemande de 3\NT est naturelle.

Il n'y a plus de place pour explorer le chelem.  S'il veut jouer un chelem, le plus simple consiste à fixer l'atout au niveau de 5 par l'enchère de 5\C ou de 5\P. L'ouvreur ne devrait pas avoir du mal à juger de la valeur de sa main.

\end{multicols}

\titre{2\C--2\NT}

\begin{multicols}{2}

Avec 12-13H et une distribution raisonnable, l'ouvreur passe.
Avec 14H, l'ouvreur peut annoncer 3\P dans 3 cartes car le répondant pourrait y avoir 5 cartes. De même, il peut annoncer sa 6\ieme carte à \C
Sinon, il peut conclure à 3\NT. Dans cette zone, on n'a pas les moyens de s'enquérir des arrêts.

Les enchères de 3\T et de 3\K, encourageantes mais non forcing, montrent des distributions vraiment irrégulières disqualifiant le contrat de 3\NT.
\end{multicols}

