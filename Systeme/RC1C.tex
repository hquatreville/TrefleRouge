\chapter{ L'ouverture de 1\C}

L'ouverture de 1\C indique une main d'au moins 15H sans majeure quatrième.


Expliquons la réponse de 1\P. C'est une réponse qui dénie 5 cartes à \P. C'est important pour la suite des enchères. Après cette enchère de 1\P, toute enchère à pique devient artificielle et permet d'annoncer toutes les mains qui ne rentrent pas dans les cases. On répond 1\P avec toutes les mains faibles de 0 à 5 H (sans 5 cartes à \P bien sur). Et répond aussi 1\P à partir de 9H ce qui dans le contexte est forcing de manche. Avec 5 cartes à \C, dans cette zone faible, on répond aussi 1\P.

La logique de la réponse de 1\P est donc a peu près la même que sur ouverture de 1\T ou de 1\K, à ceci près qu'elle dénie 5 cartes à pique.

Avec 5 cartes à \P et un jeu faible, il répond 2\T non forcing. A partir de 6H, il répond 2\K, forcing. il annoncera les \C éventuels plus tard si nécessaire.



\enchbox{Ouverture de 1\C}
{
1\P&0-5H ou 9H+& Pas 5 cartes à \P \\
1SA& 6-8H & Pas de majeure cinquième\\
2\T & 0-5H & 5 cartes à \P \\
2\K & 6H+ & 5 cartes à \C \\
2\C & 6H+& 5 cartes à \P \\
2\P & 3-8H & 6 belles cartes à \P \\
2\NT & 9H+ & 6 belles cartes à \T \\
3\T & 9H+ & 6 belles cartes à \K \\
}




\titre{1\C--1\P}

L'ouvreur fait une enchère en face d'une main potentiellement ultra-faible.
Si le répondant est fort, il reparlera et l'ouvreur avisera.

Comme dans tous le système, les enchères à saut de l'ouvreur (sauf 2\NT\ 22-23H) sont forcing de manche.
Avec une vingtaine de points seulement, l'ouvreur utilise artificiellement les enchères de 2\C et 2\P.

Les valeurs en point H sont indicative. Il faut plutôt compter en levées de jeu.



\enchbox{1\C--1\P}
{
1\NT & 15-19H & régulier \\
2\T  & 15-19H & naturel \\
2\K  & 15-19H & naturel \\
2\C  & 19H+ & bicolore mineur\\
2\P  & 19-22H & unicolore mineur\\
2\NT & 22-23H & régulier\\
3\T &22H+& naturel forcing de manche\\
3\K &22H+& naturel forcing de manche\\
3\NT && pour les jouer \\
}


\titre{1\C--1\P--1\NT}

Comme dans toutes les séquences 1x--1\P--1\NT, les enchères au niveau de 2 indiquent une misère, l'enchère de 2\NT est quantitative et les enchères au niveau de 3 montrent un singleton.

\enchbox{1\C--1\P--1\NT}
{
2\T && conclusion\\
2\K && conclusion\\
2\C && conclusion \\
2\P &9H+& bicolore mineur\\
2\NT &14H+& quantitatif \\
-> & 3\T & 15H \\
\rb -> & 3\K & 16H \\
-> & 3\C & 17H \\
\rb -> & 3\P & 18H \\
-> & 3\NT & 19H \\
3\T && singleton \T \\
3\K && singleton \K \\
3\C && singleton \C \\
3\P && singleton \P \\
3\NT&9-13H& conclusion
}


\titre{1\C--1\P--2\NT}

Comme sur toute redemande à 2\NT, 3\T est un checkback Stayman, à la recherche d'un fit \C (le fit \P étant exclu par la réponse de 1\P);
Les autres enchères annoncent un singleton.

Il n'est hélas pas possible d'annoncer un singleton \T sans dépasser 3\NT. Avec un tel singleton, il est toujors possible d'annoncer 4\K en cas d'espoir de chelem. Sinon 3\NT fera l'affaire.

\enchbox{1\C--1\P--2\NT}
{
3\T && checkback Stayman \\
\rw -> &  3\K & impossible \\
\rw -> &  3\C & 3 cartes à \C \\
\rw -> & 3\P & impossible \\
\rw -> & 3\NT & 2 cartes à \C \\
3\K && singleton \K \\
3\C && singleton \C \\
3\P && singleton \P \\
3\NT && conclusion \\
4\T && chelem \T \\
\rb -> & 4\NT & refus \\
\rw 4\K && chelem \K \\
 -> & 4\NT & refus \\
}

\textit{Remarque : Les esprits tordus peuvent modifié le check-back pour y inclure la possibilité du singleton \T. L'effort de mémoire à fournir rend la chose scabreuse.}


\titre{1\C--2\K}

Le répondant montre 5 cartes à \C dans un jeu pas complètement nul. Avec une main banale et 2 ou 3 cartes à \C, jusqu'à 17H, l'ouvreur se contente de rectifier à 2\C. Tous les sauts sont forcing de manche. Avec un jeu fort sans enchère naturelle, on utilise le Joker à 2\P.



\enchbox{1\C--2\K}
{
2\C & 15-17H & 2 ou 3 cartes à \C \\
2\P & 18H+   & non fitté, forcing manche \\
2\NT & 18-19H & régulier \\
3\T  & 15-17H & naturel \\
3\K  & 15-17H & naturel \\
3\C  & 18H+ & fit forcing de manche \\
}

