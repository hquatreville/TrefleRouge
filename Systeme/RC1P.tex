\chapter{ L'ouverture de 1\P}

Pour les habitués de la majeure cinquième, l'ouverture de 1\P proposée ici est fort différente.

La première différence est la limitation à 17H. La seconde différence est l'exclusion des mains 5-3-3-2 du système.
Le système de réponses proposé ici obéit au cahier des charges suivants : rester le plus naturel et le plus simple possible, rester le plus cohérent avec le reste du système.

En conséquence, comme pour les autres ouvertures au niveau de 1, les changements de couleur à saut à 3\T, 3\K et 3\C sont des propositions naturelles non forcing et sur les enchères Texas de 2\T, 2\K et 2\C, l'ouvreur rectifie le Texas avec une main banale ou dévoile ses batteries avec une main forte.

Le répondant, avec une main de 0 à 7H, doit en premier lieu songer à passer.  Il faut passer, particulièrement avec deux ou trois atouts. Bien sur, les adversaires vont réveiller. Mais les réveils après une ouverture de 1\P sont inconfortables. Les laisser se dépêtrer peut conduire à de bonnes surprises.

Avec 4 atouts, il est moins astucieux de passer mais on n'a pas vraiment le choix sauf à faire un barrage à 4\P en fonction de la vulnérabilité.

Les seules mains faibles où il est habile de continuer le dialogue sont les mains avec un singleton pique. Dans ce cas, le relai à 1\NT a de bonnes chances d'améliorer la partielle.

A partir de 8H ou 9DH, le répondant peut montrer son soutien à pique de 4 manières. Avec 5 atouts, il peut faire un splinter ou un barrage. Avec 4 atouts, il peut faire un mixted raise à 3\P. S'il est certain de jouer la manche sans ambition de chelem, il donne le soutien par l'intermédiaire de l'enchère de 2\NT et l'ouvreur conclue à 4\P


Dans tous les autres cas, il donne le soutien en Texas à 2\C. Cette façon de faire en Texas laisse suffisamment d'espace pour développer toutes les mains limite de manche ou de chelem.
La seule chose importante à retenir : \textit{Ne jamais faire de soutien différé} ! Toute autre enchère que 2\C dénie un fit à pique.


Avec 5 cartes à \C (et 8H, cela va sans dire), le répondant fera un Texas à 2\K. Ici encore, la main n'est pas zonée mais la souplesse du Texas permet de poursuivre le dialogue sans difficultés.

A partir de 13H, sans 3 cartes à pique et sans 4 cartes à  à \C, on fera un relai à 2\T. Utiliser l'enchère de 2\T comme relai poubelle forcing de manche n'est pas une idée nouvelle !
Le répondant peut être unicolore \T, unicolore \K, ou même bicolore mineur.
Sur ce relai à 2\T, la redemande poubelle est 2\K (et non pas 2\P) comme si 2\T était un Texas. C'est un puppet.
Cette façon de faire est économique et cohérente, ainsi on répond de la même façon aux enchères de 2\T, 2\K et 2\C.

Il reste donc 3 enchères pour les mains de 8 à 12H sans 5 cartes à \C et sans 3 cartes à \P.

Avec 8-9H, exactement 2 cartes à \P, et sans 4 cartes à \C, on répond 2\P. C'est une préférence anticipée. Avec ces mains, on aurait rectifié à 2\P. Autant le faire tout de suite. De temps en temps, on  ratera un fit neuvième en mineur au profit d'un fit septième en majeure. C'est le prix à payer. Et lorsque les adversaires gagnent 3\C, c’est tout bénéfice. Trouver notre fit mineur les aurait aider à trouver le leur.

L'enchère la plus souple est la réponse à 1\NT. Elle peut être faible, pas nulle quand même, avec un singleton \P. Avec 4 cartes à \C, elle commence à 8H et n'a pas de limite supérieure. Avec deux cartes à \P et sans 4 cartes à \C, elle va de 10 à 12H.


\enchbox{1\P}
{
1\NT & 5H+ & Dénie 3 cartes à \P. Forcing, illimité si 4 cartes à \C \\
2\T  & 12H+ & Forcing de manche, pas de fit, pas 4 cartes à \C \ \\
2\K  & 8H+  & 5 cartes à \C \\
2\C  & 9DH+ & 3+ cartes à \P \\
2\P  & 8-9H & 2 cartes à \P, préférence anticipée  \\
2\NT & 12-15H & Pour jouer 4\P \\
3\T  & 9-11H & 6 belles cartes \\
3\K & 9-11H & 6 belles cartes \\
3\C & 9-11H & 6 belles cartes \\
3\P & 9-10DH & 4 cartes \\
3SA &7-11H& Chicane indéterminée et 5 atouts \\
4\T &7-11H& Splinter 5 atouts \\
4\K &7-11H& Splinter 5 atouts \\
4\C &7-11H& Splinter 5 atouts \\
4\P &0-10H& Barrage 5 atouts \\
}




\titre{1\P--2\P}

Danger : l'enchère promet deux cartes et non pas trois.

Le répondant possède 8-9H. L'ouvreur passe le plus souvent. Il peut conclure à 4\P avec 6 cartes ou à 3\NT avec 16-17H.

Il peut montrer sa deuxième couleur à la recherche d'un arrêt dans son singleton s'il veut jouer 3\NT. Mais il faut une grosse distribution pour procéder ainsi. En points d'honneur, la manche mineure est vraiment lointaine.

S'il hésite, il peut toujours proposer 2\NT ou 3\P.


\titre{1\P--2\C}

\begin{multicols}{2}



La réponse de 2\C annonce un fit, à partir de 8H mais peut provenir d'une main très forte. Le répondant a besoin de connaître la force de l'ouverture soit en vue de la manche, soit pour jouer un chelem.
L'ouvreur n'a que deux réponses possibles et c'est le répondant qui poursuit le dialogue en fonction de ses ambitions. La réponse de 2\NT est de facto forcing de manche. L'ouvreur se méfiera des mains de 15H avec un honneur sec.

\enchbox{1\P--2\C}
{2\P & 11-15H& \\
2\NT & 15-17H &\\
}



Les redemandes du répondant au niveau de 3 sont naturelles et demandent un complément pour jouer le chelem. 3\NT est une demande de contrôle brutale. Les splinters différés sont très fort et demandent à l'ouvreur d'aller au chelem sauf points perdus dans la couleur.

Il faut bien comprendre que la séquences 1\P--2\C--2\P--3\T et 1\P--2\C--2\NT--3\T remplacent les soutiens différés des systèmes classiques 1\P--2\T--2x--3\P. Toutefois, la séquence est à la fois plus naturelle (les trèfles sont toujours réels), plus économique et plus précise.

La redemande à 2\NT est quantitative de manche et la redemande à 3\P est quantitative de chelem.

Sur la proposition de manche à 2\NT, l'ouvreur peut revenir à 3\P, nommer la manche ou, s'il hésite, dire 3\T qui ne veut rien dire d'autre que : «Partenaire, nomme la manche avec 12 (ou 10) et revient à 3\P avec 11 (ou 9)». Dans ce contexte, inutile de raconter sa vie.



\enchbox{1\P--2\C--2\P}
{
\Pass & 9-10 DH&\\
2\NT & 11-12 DH&\\
3\T  && 5 cartes, chelem \\
3\K  && 5 cartes, chelem \\
3\C  && 5 cartes, chelem \\
3\P  &19-20DH& Quantitatif \\
3\NT &21DH+& Contrôles \\
4\T  && Splinter fort \\
4\K  && Splinter fort \\
4\C  && Splinter fort \\
4\P & 13-20 DH &\\
}

\enchbox{1\P--2\C--2\NT}
{\
3\T  && 5 cartes, chelem \\
3\K  && 5 cartes, chelem \\
3\C  && 5 cartes, chelem \\
3\P  &15-16DH& Quantitatif \\
3\NT && Contrôles \\
4\T  && Splinter fort \\
4\K  && Splinter fort \\
4\C  && Splinter fort \\
4\P & 9-16 DH &\\
}


\end{multicols}
\titre{1\P--3\P}

Le soutien n'est pas un pur barrage, c'est un mixted raise, 4 atouts et 9-10 DH. Ces mixted raise ont remplacé petit à petit les barrages faibles. C'est plus efficace et plus précis.

A priori pas de chelem en vue mais, si besoin était, du genre bicolore 6-6, avec des adversaires endormis l'ouvreur peut toujours déclencher les contrôles à 3\NT ou annoncer sa deuxième couleur.


\titre {1\P -- 1\NT}

Le relai à 1\NT est forcing un tour et invite l'ouvreur à décrire son jeu. Cette description se fait de façon économique sauf avec une main distribuée dans la zone 16-17H.

Attention, l'ouvreur est limité à 17H. Donc avec les mains de 0 à 7H, il faut en priorité songer à passer.  Avec un singleton ou une chicane pique et un jeu pas trop nul (avec une main nulle, les adversaires vont faire au moins un tentative de manche), le relai à 1\NT permettra le plus souvent de trouver une meilleure partielle.

Le plus souvent, l'enchère de 1\NT provient d'un jeu de 8 à 12H sans 4 cartes à pique.

Toutefois, l'enchère de 1\NT peut aussi provenir d'une main avec exactement 4 cartes à \C (ou 5 cartes moches dans un jeu faible). Dans ce cas, il n'a pas de limite supérieure.

\enchbox{1\P--1\NT}
{
2\T & 11-17H & 4+ cartes à \T \\
2\K & 11-17H & 4+ cartes à \K \\
2\C & 11-17H & 4+ cartes à \C \\
2\P & 11-17H & 6 cartes à \P \\
2\NT & 16-17H & 6 cartes à \P et une mineure 4\ieme \\
3\T & 16-17H & 5 cartes à \T \\
3\K & 16-17H & 5 cartes à \K \\
3\C & 16-17H & 5 cartes à \C \\
3\P & 16-17H & 7 cartes à \P ou 6 très belles\\
}


\titre{1\P--1\NT--2\T}

L'ouvreur possède 5 cartes à pique et 4 cartes à \T.

Si le répondant a un jeu faible, il passe ou redemande 2\K dans sa longueur.

Avec 8-10H, il fait une enchère de préférence, 2\P avec deux cartes à \P ou 3\T avec 4 cartes à \T et singleton \P.

Avec 11-12H, le répondant redemande 2\NT.

Avec 13H+, s'il ne souhaite pas conclure à 3\NT le répondant commence par dire 2\C montrant 4 cartes dans une belle main. Une enchère subséquente à \K de sa part doit être comprise comme naturelle, Canapé, avec des intentions de chelem. De même, une enchère à \T indiquera un fit forcing de manche.

Le plus souvent, c'est l'arrêt carreau qui manque pour jouer à \NT. De temps à autre, il y a un chelem à jouer.



Cette séquence 1\P--1\NT--2\K est similaire. La différence, c'est qu'une longueur à \T dans un jeu faible sera exprimée au niveau de 3.

\titre{1\P--1\NT--2\C}

Cette séquence est une plaie dans tous les systèmes. La main de l'ouvreur va jusque 17H. Le répondant se sent dans l'obligation de parler à partir de 8H. Mais l'ouvreur, ne sachant pas s'il en a 8 ou 10 doit tenter le coup à pile ou face. Avec le deux sur un forcing de manche, la situation est encore pire puisque la main du répondant va jusque 11H.

En trèfle rouge, la main du répondant est illimité mais \dots\ on a une solution. L'enchère de 2\P dans cette séquence est artificielle ! Elle promet un fit \C d'au moins 10H.


\enchbox{1\P--1\NT--2\C}
{
Passe & 0-7H & 3+ cartes à \C \\
2\P & 10H+ &  4 cartes à \C \\
2SA & 10-12H & Régulier\\
3\T & 7-8H & Naturel \\
3\K & 7-8H & Naturel \\
3\C & 8-9H & 4 cartes à \C
}

\enchbox{1\P--1\NT--2\C--2\P}
{
2\NT & 15-17H& Belle main \\
3\C &11-12H& \\
4\C & 13-14H& \\
}


\titre{1\P--2\T}

La réponse à 2\T impose la manche. La possibilité d'un fit \P existe si l'ouvreur possède 6 cartes. La possibilité d'un fit \C existe si l'ouvreur possède 5 cartes.
En aucune façon, l'ouvreur ne nomme des cœurs 4\ieme.

\enchbox{1\P -- 2\T}
{
2\K &11-17H & RAS \\
2\C &11-17H& 5+ cartes à \C \\
2\P &11-17H& 6+ cartes à \P \\
2\NT & 16-17H & 6 cartes à \P et une mineure 4\ieme \\
3\T & 16-17H & 5 cartes à \T \\
3\K & 16-17H & 5 cartes à \K \\
3\C & 16-17H & 5 cartes à \C \\
3\P & 16-17H & 7 cartes à \P ou 6 très belles\\
}

La deuxième enchère du répondant est naturelle. Toutefois, il faut comprendre que le répondant a dénié 4 cartes à \C. Toute enchère à \C de sa part doit être considérée comme artificielle sans enchère naturelle. L'enchère remplace une 4\ieme forcing.
