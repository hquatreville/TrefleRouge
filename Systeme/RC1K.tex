\chapter{L'ouverture de 1\K}

\begin{multicols}{2}


L'ouverture de 1\K promet 4 cartes à pique.

Elle ne peut pas comporter exactement 4 cartes à \C (on ouvrirait de 1\T).

Dans la zone 12-14H, la main peut avoir des cœurs longs qui seront annoncés en Canapé au tour suivant.

La main peut posséder cinq cartes à pique soit avec une main 5-3-3-2 de 15-17H, soit toute distribution à partir de 18H.

Le répondant dispose de deux relais, le relai à 1\C avec des mains de 5 à 11H, toute distributions et le relai à 1\P, soit avec des mains très faibles (moins de 5H) ou faible (jusqu'à 7DH) et fittées à \P, soit avec des mains plates non fittées et forcing de manche.

Les enchères de 1\NT, 2\T et 2\K sont des Texas forcing de manche. l'enchère de 2\C est Texas aussi, elle indique un fir limite de manche.


\enchbox{Ouverture de 1\K}
{
1\C & 5-11H & Pas 4 cartes à \P \\
1\P & 0-4H & Toute distribution \\
\rb & 5-7DH & 4 cartes à \P \\
& 12H+ & Pas 4 cartes à \P \\
1\NT & 12H+ & 5 cartes à \T \\
2\T & 12H+ & 5 cartes à \K \\
2\K & 12H+ & 5 cartes à \C \\
2\C & 11DH+ & 4 cartes à \P \\
2\P & 8-10DH & 4 cartes à \P \\
2\NT & 12-15H & 4(5) cartes à \P sans singleton\\
3\T & 9-11H & 6+ cartes à \T \\
3\K & 9-11H & 6 cartes à \K \\
3\C & 9-11H & 6+ cartes à \C \\
3\P & 7-9H & 5 cartes à \P \\
3\NT & 8-11H & Chicane indéterminée \\
4\T & 8-11H & Singleton \T\\
4\K & 8-11H & Singleton \K\\
4\C & 8-11H & Singleton \C\\
4\P & 0-10H& Barrage 6 cartes \\
4\NT && Blackwood brutal\\
}

\end{multicols}

\titre{1\K--1\C}

Le répondant a au plus une main limite de manche. De plus il n'a pas 4 cartes à \P.

Le plus souvent, on va jouer 1\NT.
La procédure est la même que sur l'ouverture de 1\T. La redemande de l'ouvreur à 1\NT indique une main régulière de 15 à 17H. Avec un jeu régulier de 12-14H, il utilise l'enchère Texas de 1\P de sorte que la main sera joué du même coté que le champ.

Les enchères sans saut sont naturelles et non forcing. A défaut d'avoir trouvé un fit à pique, on cherche un fit dans la vrai couleur.

Mise à part 2\NT, les enchères à saut sont naturelles et forcing de manche. Toutes les mains fortes non forcing de manche passe par l'enchère de 1\P. Dans la plupart des cas, le répondant annoncera 1\NT et l'ouvreur poursuivra de façon naturelle non forcing.

Ce chapitre est moins détaillé que le chapitre sur l'ouverture de 1\T car, en réalité, tout ce qui est nécessaire pour développer l'ouverture de 1\K n'est qu'un sous-ensemble de ce qui est nécessaire pour développer l'ouverture de 1\T. Se pose quand même le problème des fit \C quand le répondant est dans la zone 5-11H.

Avec 5 cartes à \P, l'ouvreur doit annoncer 1\NT dans la zone 15-17H. Avec un jeu plat de 18H, il peut répondre 2\NT. Sinon, la réponse de 1\K étant positiven, il impose la manche par l'enchère de 2\P.

\enchbox{1\K--1\C}
{
1\P & 12-14H & Régulier\\
\rb& 22H+ & Régulier\\
& 15H+ & Toute distribution \\
1\NT & 15-17H & Régulier \\
2\T & 12-14H & 5 cartes à \T \\
2\K & 12-14H & 5 cartes à \K \\
2\C & 12-14H & 5 cartes à \C \\
2\P & &5+ cartes, forcing de manche \\
2\NT & 18-19H & Régulier \\
3\T  && Canapé forcing de manche \\
3\T  && Canapé forcing de manche \\
3\C && 5-5 forcing de manche \\

}

\titre{
  1\K \\
  Réponses après intervention
}

Les Texas mineurs perdent leurs caractères forcing de manche et commencent à 8H.

\section*{Intervention par contre}

Le \Redouble est une enchère punitive. Il dénie 4 cartes à \C et promet 10H et deux couleurs 4\ieme. Tous les contre subséquents sont punitifs.

Avec un fit \P, on donne le fit à 2\P ou on fait un Texas, le tout en ignorant de \Double. L'intervention donne une troisième façon de donner le fit, l'enchère de 1\P avec un jeu de manche plat sans ambition de chelem. Attention, cette façon de faire fait jouer le coup de l'autre main.

\section*{Intervention par 1\C}

Le \Double et l'enchère de 1\P montrent a peu près le même type de jeu. Une main non fittée, avec ou sans l'arrêt  \C (dans tous les cas, on préfère faire jouer l'ouvreur). Le \double est plutôt compétitif, en tout cas limité à 11H alors que l'enchère de 1\P est forcing de manche. Par ailleurs, l'enchère de 1\P peut masquer une main fittée de manche dans un jeu plat.

\titre{1\K--1\C--1\P}
Dans cette séquence, le répondant considère, dans un premier temps que l'ouvreur possède 12-14H. Retrouver les fit \C est important.

Dans cette séquence, le répondant n'a pas le droit de dépasser 2\NT pour laisser à l'ouvreur la possibilité de développer une main forte.

\enchbox{1\K--1\C -- 1\P}
{
1\NT & 5-10H & Régulier\\
2\T & 5-10H & (5)6 cartes \\
2\K & 5-10H & (5)6 cartes \\
2\C & 5-10H & 5+ cartes à \C \\
2\P & 10-11H & 5 cartes à \C \\
2\NT & 11H & Régulier\\
}

\titre{1\K--1\C--1\NT}

\enchbox{1\K--1\C -- 1\NT}
{
2\T & 5-10H & (5)6 cartes \\
2\K & 5-10H & (5)6 cartes \\
2\C & 5-7H & 5+ cartes à \C \\
2\P & 8-11H & 5 cartes à \C \\
2\NT & 8H & Régulier\\
3\T & 9-11H & Singleton \T \\
3\K & 9-11H & Singleton \K \\
3\C & 9-11H & Singleton \C \\
3\P & 9-11H & 3 cartes à \P\\
}

\titre{1\K--1\C--2\NT}
\enchbox{1\K--1\C -- 2\NT}
{

3\T &  & Check-back \\
-> & 3\K & 3 cartes à \C et 5 cartes à \P\\
-> & 3\C & 3 cartes à \C et 5 cartes à \P\\
-> & 3\P & 2 cartes à \C et 5 cartes à \P\\
-> & 3\NT & 2 cartes à \C et 4 cartes à \P\\
3\K &  & Singleton \K \\
3\C &  & Singleton \C \\
3\P &  & Singleton \T\\
}

\titre{1\K--1\P}

\enchbox{1\K--1\P}
{
1\NT & 12-19H & Régulier\\
2\T & 12-21H & Canapé \\
2\K & 12-21H & Canapé \\
2\C & 12-21H & Canapé \\
2\P & 18-21H & 5+ cartes à \P \\
2\NT & 22-23H & Régulier \\
3\T & & Canapé forcing de manche \\
3\K && Canapé forcing de manche \\
3\C && 5-5 forcing de manche \\
3\P && Naturel forcing de manche \\
}

\titre{1\K--1\P--1\NT}

\enchbox{1\K--1\P -- 1\NT}
{
2\T & 0-4H & (5)6 cartes \\
2\K & 0-4H & (5)6 cartes \\
2\C & 0-4H & 5+ cartes à \C \\
2\P & 0-7DH & 4 cartes à \P \\
2\NT & 14H+ & Quantitatif\\
-> & 3\T & 12-13H\\
-> & 3\K & 14-15H\\
-> & 3\C & 16-17H\\
-> & 3\P & 18-19H\\
}

\titre{1\K--1\P--2\NT}
\enchbox{1\K--1\P -- 2\NT}
{

3\T &  & Check-back \\
-> & 3\K & 3 cartes à \C et 5 cartes à \P\\
-> & 3\C & 3 cartes à \C et 5 cartes à \P\\
-> & 3\P & 2 cartes à \C et 5 cartes à \P\\
-> & 3\NT & 2 cartes à \C et 4 cartes à \P\\
3\K &  & Singleton \K \\
3\C &  & Singleton \C \\
3\P &  & Singleton \T\\
}
