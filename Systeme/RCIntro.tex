\chapter*{Introduction}

\section{Pourquoi un nouveau système ?}

La théorie des enchères avance très vite.

Depuis la parution du best-seller de Pierre Jaïs et Michel Lebel en 1976 intitulé «La majeure cinquième» puis en 1982 «La nouvelle majeure cinquième» sous-titrée pompeusement «Le Bridge standard français», la majeure cinquième occupe une place de plus en plus hégémonique en France. On pourrait croire que ces ouvrages, de grande qualité, en sont la cause. Mais il n'en est rien puisque dans le monde entier, la majeure cinquième est devenue standard.

Et il y a une raison à cela. Avant la majeure cinquième, on jouait le plus souvent un système de type ACOL, nommé «La longue d'abord». Ce système est tombé aux oubliettes car à la fois plus difficile à manipuler et moins performant que la majeure par cinq. Toutefois, on lui doit beaucoup. En effet, conscient du caractère perfectible de ce système, beaucoup de champions ont développés des idées nouvelles. Tout particulièrement, un système nommé «Le Canapé» où, dans certaines situations, on annonce les bicolores en commençant par la couleur la plus courte. Le concept est resté ! Et plus important, le trèfle fort, un système qui consiste à ouvrir les mains fortes de 1\T ce qui permet de mieux zoner les autres ouvertures et, cerise sur le gâteau, de se débarrasser de l'ouverture de 2\T forcing de manche. Eric Rodwell, sept fois champion du monde, c'est un record, joue un système de trèfle fort.

De toute cette effervescence, c'est la majeure par cinq qui a émergé. j'ai relu l'ouvrage de Jaïs et Lebel de 1982. Tout est clair. Pas de gadgets. Juste quatre conventions en enchères à deux, le Blackwood, le Stayman, le Texas et la quatrième couleur forcing ; Et deux conventions, le contre d'appel et le contre Spoutnik en enchères à quatre. Et un principe simple : Quand on est fort, on fait un saut ou, à défaut, on annonce une nouvelle couleur.

Regardons comment a évolué le système français en 42 ans pour donner le SEF 2024. De nombreuses conventions se sont greffées au système.

Commençons par le Roudi. Le Roudi est un gadget, c'est à dire une enchère interrogative avec un système de réponses codifiées permettant à l'instigateur de prendre la bonne décision. Il entre en troisième position après le Blackwood et le Stayman. Ces gadgets sont très populaires. On apprend le système de réponse set on se débrouille. De même que le Stayman ou le Blackwood, le Roudi est un gadget indispensable dans tous les systèmes de majeure par cinq. Sans lui, on a un problème pour zoner les mains. Comme le Stayman, comme le Blackwood, le systèmes de réponses a évolué au fil du temps.

Le Roudi est important dans la théorie des enchères car il met bien en lumière comment un système se construit. On impose les premières enchères. On regarde ce qui se passe. Et quand on voit un trou dans la raquette, on ajoute un gadget. Au final, le joueur de club joue avec 4 ou 5 gadget couvrant les situations les plus fréquentes et le joueur de compétition, plus exigeant, va jouer avec une dizaine de gadget de façon à couvrir des situations moins fréquentes. Mais comme il joue souvent, cela revient grosso modo au même.

Il y a des alternatives au Roudi. Citons le double deux et le ping-pong. Ces deux conventions, à peu près équivalentes au Roudi en terme d'efficacité, sont terriblement impopulaires. Pourquoi ? Ce ne sont pas des gadgets ! D'ailleurs, en français, il n'y a pas de nom pour qualifier ces techniques. En anglais, on appelle cela des \textit{puppets} car le partenaire agit comme une marionnette. Le répondant commence par dire 2\T, l'ouvreur rectifie à 2\K (comme une marionnette) et le répondant s'explique. En français, le mot \textit{puppet}
est improprement utilisé pour décrire une façon particulière de jouer le Stayman. Bref, cette technique de \textit{puppet} est marginalisée. En effet, on transfert la décision au partenaire. Or comme chacun sait, le partenaire n'est pas digne de confiance \dots\ en général ! Bref, les gadgets sont beaucoup moins toxiques pour l'ego des joueurs que les puppets et ce n'est pas négligeable.

La deuxième convention que je souhaite examiner ici est le 2\NT fitté. C'est à mon sens, la plus grande révolution des enchères en 42 ans. Ce n'est pas un gadget ! C'est une enchère de moussaillon. Une enchère de moussaillon est une enchère où on vend sa main en une seule fois. On laisse alors son partenaire se débrouiller et on constate avec délectation que, contrairement à tous les pronostics, le brave bougre prend systématiquement les bonnes décisions. Le SEF 2024 regorge d'enchères de moussaillon, citons les enchères de rencontre ou les splinters.

Ce n'est pas la convention qui est importante mais ce sont ses conséquences. Avant le 2\NT fitté, le plus gros cauchemar des bridgeurs était de s'entendre sur la signification \textit{forcing } ou \textit{non forcing} de telle ou telle séquence. En 1984, Michel Perron, champion du monde, a publié un ouvrage de 57 pages intitulé «Que jouons-nous partenaire ?». Dans cet ouvrage il y a, entre autre, 32 séquences dont il faut préciser le caractère \textit{forcing} ou non en accord avec chaque partenaire. Une bonne école, mais fastidieuse.
Dans le SEF 2024, il suffit d'une demi-page pour expliquer ce qui est \textit{forcing}. En gros, tout ce qui ressemble à une enchère \textit{forcing} est une enchère \textit{forcing} ! Avec une unique exception, la séquence 1\C--1\P--2\C--3\C. Quel confort ! Et le deux sur un forcing de manche va dans le même sens. Moins d'exceptions, plus de confort. Quant à la séquence 1\C--1\P--2\C--3\C, on  ne peut pas la jouer \textit{forcing} \dots\ sans ajouter un gadget, par exemple en utilisant 2\P de façon conventionnelle; il n'y a donc pas de gain de simplicité à le faire.

Toujours est-il que les enchères évoluent de deux façons différentes et opposées. Ves une plus grande efficacité grâce à l'adoption de gadgets.
Chaque gadget permettant de combler  ou d'atténuer un inconvénient du système. Mais ce gain est coûteux. Il faut apprendre le système de réponses de ces gadgets. Et comme ces systèmes évoluent dans le temps, il faut vérifier qu'on a le même que le partenaire. Il y a même des systèmes de réponses qui servent de marqueurs sociaux. Typiquement le Blackwood 41-30 au lieu de 30-41. En région lyonnaise par exemple, jouer le Blackwood 30-41 vous disqualifie aux yeux des joueurs de compétitions car «comme chacun sait, le 41-30 est meilleur !».
Et je ne parle pas de la quatrième et de la troisième couleur forcing qui, une fois sur dix, mènent à des situations vraiment difficiles !
Le joueur malin, n'en a cure, quand la situation lui échappe, il propose 3\NT, au pire, il chutera comme les autres.

Les meilleures innovations sont celles qui amènent de la simplicité. Mais ce sont les plus difficiles. Il y a beaucoup plus d'intelligence derrière le 2\NT fitté que derrière le Roudi. Non seulement, les innovations simplificatrices sont difficiles à concevoir, mais pire, il est difficile de convaincre les bridgeurs de leurs intérêts. En effet, les bridgeurs ayant mis des années à maîtriser une méthodes complexes ne voient pas l'intérêt de passer à quelque chose de plus simple mais qui les oblige à changer leurs habitudes. Ils préfèrent, et de loin, le dernier gadget à la mode, préconisé par leur gourou local. Et c'est ainsi que le deux sur un forcing de manche a du mal à s'imposer. Et l'ouvrage d'Alain Levy, «Le Système Y» ne va certainement pas améliorer la situation. Ce qu'Alain Levy propose est beaucoup trop avancé. Il propose une méthode de champion du monde \dots\ pour champion du monde. Terriblement efficace et terriblement compliquée. Tous les gains en simplicité du nouveau système sont absorbés par l'ajout de petits gadgets, certes utiles et intelligents, voire fortement optimisés, mais qui rendent difficile l'apprentissage du système dans son ensemble.
Pour que les bridgeurs adoptent le deux sur un, il faut leur proposer un système basique, plus simple que le SEF, déjà très avancé, et aussi efficace que celui-ci. Ensuite, dans un deuxième temps, et pour ceux qui veulent, on ajoute les gadgets petit à petit.

En tout cas, aussi bien le 2\NT fitté que le deux sur un forcing de manche font apparaître une vérité. Une loi ! Un peu comme la loi des levées totales. Pour simplifier un système d'enchère, il faut le modifier en amont, pas en aval. En aval, on ajoute des gadgets. Et chaque gain en efficacité se fait au prix de la complexité. Par contre, une modification en amont peut amener soit de la simplicité à efficacité égale, soit de l'efficacité à complexité égale.

Oui mais comment faire les changements nécessaires ?

L'idée fondatrice qui a présidé à l'élaboration du Trèfle Rouge consiste à aligner le système d'enchères sur la stratégie des enchères. Je vais m'expliquer sur ce point car c'est la base de tout.

On trouve sur la toile un site amusant et très conne en France «Bridge Academy». Toute les semaines, le site examine une séquence d'enchère et vend un petit polycopié sur le sujet. A chaque fois, il rappelle la stratégie des enchères. Et il applique cette stratégie à la séquence
étudiée pour en déduire la conduite à tenir. Dans la plupart des cas, on va trouver un ou deux gadgets permettant de gérer les situations les plus épineuses. Avec une séquence par semaine, il faut environ 4 ans pour avoir un système à peu près complet. Au bout de ces 4 ans, les enchères ont fait des progrès et on réexamine les séquences. Ce qui garanti des revenus réguliers aux auteurs du site et aide beaucoup de bridgeurs de compétition.

Mais pourquoi attendre la troisième ou la quatrième enchère ? Pourquoi ne pas appliquer la stratégie des enchères dès la première enchère ?

Si on applique la stratégie des enchères dès la première enchère, on va avoir des séquences d'enchères plus courtes. Une séquence plus courte a trois avantage. Elle est plus simple. Elle en dévoile moins aux adversaires, rendant le flanc plus difficile. Et elle laisse moins de place pour intervenir.

\section{La stratégie des enchères}

Il y a deux types de séquences d'enchères. Les séquences compétitives et les séquences à deux.

Les séquences compétitives peuvent être symétriques, les deux camps ont entre 18 et 22H ou elles peuvent être dissymétriques avec un camp en attaque et un camp en défense.

Dans les séquences compétitives symétriques, il faut trouver son fit le plus vite possible. Évaluer la palier de «sécurité» distributionnelle et atteindre celui-ci en un minimum d'enchères. Il faut essayer de faire en sorte que ce soit l'adversaire qui prenne la dernière décision. En effet, celui qui prend la dernière décision peut se tromper.

Dans les séquences compétitives dissymétriques, il y a une difficulté supplémentaire. Il faut évaluer si on est en zone de manche, auquel cas, il faut la nommer même au delà du palier de «sécurité».

Dans les séquences à deux, la stratégie est plus compliquée. Il faut gérer à la fois la force et la distribution. Il faut évaluer la force combinée des deux mains pour savoir si on est en situation de partielle de manche ou de chelem.

En situation de partielle, il faut trouver un fit le plus vite possible, de préférence en majeure. En absence de fit, il faut jouer 1\NT ou en 5-2 au niveau de 2. Jouer 2\NT reste une possibilité faute de mieux.

En situation de manche, il faut trouver un fit en majeure. A défaut, il faut vérifier que 3\NT est jouable. Pour le savoir, il faut identifier que chaque joueur possède un solide arrêt dans le singleton du partenaire. Sans singleton on joue 3\NT. Si 3\NT est injouable, on tente une manche en mineure, ou à défaut dans un solide (avec beaucoup d'honneurs) semi-fit majeur.

En situation de chelem, il faut trouver son meilleur fit ou à défaut jouer 6\NT.

\section{La logique du Trèfle Rouge}

En Trèfle Rouge, on annonce toujours les majeures avant les mineures, même lorsque ces dernières sont plus longues.

On a donc besoin de 4 enchères pour décrire les majeures. On ouvre de 1\T avec 4 cartes à \C, on ouvre de 1\K avec 4 cartes à \P, on ouvre de 1\P avec 5 cartes à \P et de 2\C avec 5 cartes à \C. Ainsi, on trouve les fits 4-4 et les fits 5-3 au premier tour d'enchère.

Sans majeure, on ouvre de 1\NT, de 2\T ou de 2\K qui sont des enchères naturelles entre 12 et 14H et on ouvre de 1\C avec les jeux forts.

Ainsi, la logique du trèfle rouge est en accord avec la stratégie des enchères, trouver en priorité un fit majeur.

On ne nommera une mineure longue que pour une des trois raisons suivantes : Pour trouver la meilleure partielle et la mineure sera alors annoncée en Canapé. Pour éviter un mauvais 3\NT à cause d'un singleton non gardé. Et enfin pour jouer éventuellement un chelem.

Lorsqu'on trouve un fit mineur au niveau de 3, l'enchère qui suit est soit 3\NT, soit l'annonce d'un singleton, soit le déclenchement des enchères de chelem.

Voila, c’est tout.

Juste un dernier détail. On ne dispose plus d'enchère forte à 2\T ou 2\K. En conséquence, le trèfle rouge utilise la logique des trèfles forts dans une version plus moderne. Il y a deux groupes d'ouvertures. D'une part les ouvertures limitées 1\P, 2\T, 2\K et 2\C. L'ouverture 1\P, plus économique, est limitée à 17H. Les autres, au niveau de 2, sont très précises, elle vont de 12 à 14H. Ici encore, on aligne la logique du  système sur la stratégie des enchères : plus une enchère est chère, moins elle doit être fréquente. Et d'autre part, les ouvertures forcing 1\T, 1\K et 1\C qui contiennent en leur sein les mains forcing de manche, respectivement avec des \C, avec des \P et sans majeures.

Le traitement différent des \C et des \P peut donner une impression de complexité accrue. En réalité, il en est de même en majeure cinquième. Les \C et les \P posent des problèmes différents. A ce titre, il suffit de se rappeler de la fameuse séquence 1\C--1\P--2\C.

Un dernier détail. Quand on lit les réponses aux question d'Alain Levy sur le site «Bridge Academy» ou qu'on écoute les vidéos de Marc Kerlero sur son site «Amour du Bridge», on constate que ces deux auteurs, qui sont des champions pour gérer les différentes zones d'ombre de la majeure cinquième, utilisent une solution récurrente pour gérer les situations délicates. Cette solution universelle, connue de tous, est le Texas ou plus précisément le \textit{Puppet}, c'est à dire un Texas nébuleux suivi d'une clarification. Typiquement ces auteurs propose de jouer Texas ou \textit{Puppet} les séquences suivantes 1\P--1\NT--2\T--2\K (garantissant 5 cartes à \C pour l'un (Texas) et garantissant 5 cartes à \C ou 10H pour l'autre (Puppet)), 1\T--1\P--2\NT--3\C (pour pouvoir s'arrêter à 3\P d'une part et quand même pouvoir prospecter un chelem d'autre part). Et ce ne sont que les exemples les plus utiles.

En trèfle rouge, en réponses à l'ouverture, on utilise beaucoup de réponses en Texas. Et -- miracle ? -- cela permet de nettoyer beaucoup de séquences.

Au final, j'ai essayé de faire en sorte que le système soit le plus simple possible. Je laisse le soin à d'autres, le cas échéant, d'ajouter leurs gadgets pour en améliorer l'efficacité.


\section{Les failles de la majeure cinquième}

La majeure cinquième est un système efficace et surtout bien rodé par un demi-siècle d'utilisation intensive. Depuis son adoption comme standard français, deux grandes avancées théoriques ont modifié la stratégie des enchères. La première, c'est la loi des levées totales. Cette-ci ne concerne que les enchères compétitives et, en ce sens, a eu peu d'impact sur le cœur du système. La deuxième, c'est la théorie du singleton qui facilite grandement le choix entre une manche à \NT et une manche mineure. Le SEF 2024 ne prend en compte que partiellement cette deuxième avancée. Dans certaines situations, on annonce les forces et dans d'autres les singletons.

La première zone d'ombre de la majeure cinquième est son ouverture de 2\T. Je cite un champion : «Mon gars, c'est simple ! Si tu hésites entre ouvrir de 2\T et autre chose, choisi autre chose.» La messe est dite. En TPP, les ouvertures fortes sont rares donc leur inefficacité n'est pas vraiment coûteuse.

La séquence la plus archaïque et la plus coûteuse de la majeure cinquième est à mon sens
1\P--2\T--2\K--4\P. Même un défenseur médiocre va s'accrocher à sont 10 de carreau quatrième quand son partenaire a le Valet second. La notion même de soutien différé est un héritage d'une époque ancienne où le changement de couleur était la seule option forcing à disposition. En trèfle rouge, il n'y a pas de soutien différé. A partir de 11DH, on donne le soutien en Texas. Et ensuite, si on est dans la zone du chelem, et seulement dans ce cas, on montre sa longue utile. La séquence devient 1\P--2\C(*)--2\P--4\P ou 1\P--2\C(*)--2\NT--3\T à suivre. L'enchère de 2\C est un fit en Texas. L'ouvreur répond 2\P mini (12-14 points) ou 2\NT maxi (15-17 points). Notons au passage que la séquence 1\P--2\C(*)--2\NT--3\T est plus économique que son équivalent SEF 1\P--2\T--2\K--3\P, ce qui laisse toute la place disponible pour les amateurs de gadgets. De plus, l'ouvreur est déjà zoné, il ne lui reste plus qu'à dire les si les trèfles l'intéressent ou pas.

Exit les soutiens différés. Bienvenu aux fit Texas.

Une autre séquence coûteuse du SEF 2024 est la suivante 1\K--1\C--2\K--\Pass. Cette séquence, en apparence anodine, est médiocre. Sur l'ouverture de 1\K, avec son petit doubleton \C et 14H, l'adversaire a passé tranquillement en attendant de voir ce qui se passe. Au deuxième tour, il contre vaillamment, laissant le choix à son partenaire entre les \P et les \T. En trèfle rouge, la séquence est plus rapide 2\K--\Pass, ce qui laisse les adversaires dans une situation bien inconfortable.

Pourquoi la séquence ci-dessus est-elle médiocre ? En ouvrant de 1\K, on demande au partenaire de nommer une majeure que de toute façon on ne va pas soutenir. On provoque ainsi une enchère inutile (et toute enchère inutile est nuisible, inutile pour nous, utile pour l'adversaire) à 1\C.

Je ne sais pas si j'ai réussi mais ce que j'ai essayé de faire, c'est de nettoyer au maximum le système de toutes ces enchères inutiles.

\section{Un peu d'histoire}

Pourquoi ce nom, le Trèfle Rouge, et sa version anglaise Red Club. Il y a une raison mnémotechnique, l'ouverture de 1\T promet 4 cartes à \C qui est une couleur rouge. Et une raison historique, c'est un hommage au Trèfle Bleu, Le Blue Team Club une référence absolue en matière de trèfle fort. On peut télécharger la traduction anglaise de ce système italien sur le site «Clairebridge». Cet excellent système est daté, ne prenant pas en compte le principe de la majeure par cing. Il a tout de même permis au italiens de dominer la scène mondiale pendant plus de dix ans.

Les systèmes de trèfle fort ont un avantage compétitif. Ils permettent d'améliorer l'efficacité de toutes les autres ouvertures. Sans trop sacrifier les mains fortes puisque l'ouverture de 1\T, économique laisse toute la place nécessaire au développement de la main. En incorporant le principe de la majeure cinquième, les trèfles forts prennent le nom de trèfle de précision et sont toujours en vogue.

Il y a une parade aux systèmes de trèfle fort. Quand vos adversaires ouvrent de 1\T, il n'y a pas de manche à jouer dans votre camp. Vous pouvez donc intervenir de façon totalement anarchique. Cela prive l'ouvreur des paliers nécessaires pour développer son système sans vraiment les renseigner pour le jeu de la carte.

C'est cette parade qui rend impopulaire les systèmes de trèfles forts. En effet, vous passer des heures à apprendre des séquences de relai. Enfin l'ouverture chérie de 1\T arrive et vous vous apprêter à montrer à votre adversaire ce qu'est du bon bridge et celui-ci, goguenard , intervient à 1\P dans le 10 quatrième. Quelle frustration !

Pour parer à cette parade, on a développé deux astuces. Il y a le trèfle polonais. C'est un système proche du trèfle fort mais avec une ouverture de 1\NT fort. De sorte que l'ouverture de 1\T est soit 12-14H comme d'habitude, soit forte. Comme l'ouverture banale est légèrement plus fréquente que l'ouverture forte, intervenir de façon anarchique devient très risqué. La deuxième idée, c'est le système de Fantoni et Nunes. Toutes les enchères au niveau de un sont forcing et potentiellement forte et les enchères au niveau de deux limitées. De sorte que , fort ou faible, on commence toujours par nommer sa couleur.
Le prix à payer, en terme de complexité, de ce dernier système, est élevé. Les ouvertures au palier de deux, faute de place, sont assorties d'un système de relai vraiment complexe.

J'ai un peu mixer ces deux idées. En trèfles rouge, les ouvertures de 1\T et de 1\K comportent un peu plus de mains banales que de mains fortes. Quant à l'ouverture de 1\C, quand bien même promet-elle 15H, elle n'exclue pas une manche majeure chez les adversaires. Dans les deux cas, il serait contre productif pour les adversaires d'intervenir de façon anarchique. Et d'autre part, en cas d'intervention, on a déjà commencé à décrire sa main, tout particulièrement en ce qui concerne les majeures.

En résumé, dans mon esprit, le Trèfle Rouge est un descendant du Trèfle Bleu.













